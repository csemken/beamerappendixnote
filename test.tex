\documentclass{l3doc}

\usepackage{graphicx}

%% Definitions
\newcommand\fnurl[2]{\href{#1}{#2}\footnote{\url{#1}}}
\ExplSyntaxOn
\NewDocumentCommand{\example}{m m o}
  {
    \int_step_inline:nn {#2} 
      {
        \includegraphics[page=##1,width=.49\linewidth]{example-#1.pdf}
        \int_if_even:nT {##1} {\newline}
      }
  }
\ExplSyntaxOff

% TODO save to variable and use in include etc
\makeatletter
\@ifpackageloaded{tex4ht}{%
  \typeout{html}%
}{%
  \typeout{not html}%
}
\makeatother    

\begin{document}

\example{basic}{2}


\section{Usage}

\begin{function}{\appxnote}
  \begin{syntax}
   \cs{appxnote}\oarg{frameoptions}\marg{title}\marg{content}
  \end{syntax}
  
  Create a new appendix note.  Inserts a \cs{beamergotobutton} to a frame that contains \meta{content}.  The appendix frame is created using \cs{printappxnotes}.  The button’s text and the title of the appendix frame are both set to \meta{title}.  If you specify \meta{frameoptions}, these will be used when creating the beamer frame.
\end{function}

\end{document}